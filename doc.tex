\documentclass[11pt,letterpaper]{article}

\usepackage[T1]{fontenc}
\usepackage{tocbasic}

\usepackage{stat_dist}
\usepackage{xcolor}
\usepackage{tabu}

\usepackage[a4paper,%
            left=1in,right=1in,top=1in,bottom=1in,%
            ]{geometry}

\input stat_lib.tex

\usepackage{listings}
\lstset
{
    language=[LaTeX]TeX,
    breaklines=true,
    basicstyle=\tt\scriptsize,
    keywordstyle=\color{blue},
    identifierstyle=\color{magenta},
}

\setlength{\parindent}{0pt}
\setlength{\parskip}{\baselineskip}

% Redefinition of ToC command to get centered heading
\makeatletter
\renewcommand\tableofcontents{%
  \null\hfill\textbf{\Large\contentsname}\hfill\null\par
  \@mkboth{\MakeUppercase\contentsname}{\MakeUppercase\contentsname}%
  \@starttoc{toc}%
}
\makeatother

\begin{document}

\centerline{stat\_dist}

\centerline{Statistical Distribution Illustrations}

\centerline{01/17/2020 - Version 1.0 released}

\vspace*{10pt}

\centerline{\textbf{Abstract}}

The \textbf{stat\_dist} package is used to draw illustrations for
statistical distributions.  The package was written to aid in
generating pictures for an introductory statistics class
(e.g. hypothesis tests, confidence intervals, ...).  Currently, the
package has support for Normal Distributions and Student T
Distributions, with limited support for $\chi^2$ and Binomial
Distributions.

\begin{tabu} to 0.8\linewidth{|c|c|} \tabucline{-}
  \textbf{Normal Distribution} & \textbf{Student T Distribution} \\[10pt]

  \begin{NormDist}
    \begin{tikzpicture}[scale=0.6]
      \begin{axis}[DistAxis]
        \centerFill{-0.63}{-0.63}{110}{1.04}{1.04}{150}{}
        \confidenceLevel{}
        \secondAxis{125}{wpm} % mean
        \drawDist;
        \draw  node at (axis cs:-2.5, 0.3) [text width=3cm] {Area = 
          (0.5832 - 0.2643) = \textcolor{red}{0.3189}};
        \draw  [thick, -latex] (axis cs:-2.4, 0.2) -- (axis cs:-0.25, 0.1);
      \end{axis}
    \end{tikzpicture}
  \end{NormDist} &
  \begin{FakeStudDist}{9}
    \begin{tikzpicture}[scale=0.6]
      \begin{axis}[DistAxis]
        \centerFill{-1.645}{-1.645}{30}{1.645}{1.645}{50}{}
        \confidenceLevel{0.90}
        \secondAxis{40}{lbs}
        \drawDist;
      \end{axis}
    \end{tikzpicture}
  \end{FakeStudDist} \\ \tabucline{-}
   & \\
  \textbf{$\chi^2$ Distribution} & \textbf{Binomial Distribution} \\ [10pt]
  \begin{ChiDist}
    \begin{tikzpicture}[scale=0.6]
      \begin{axis}[DistAxis]
        \rightTailChi[fill=orange!50]{4}{$\chi^2$ Label}{Alpha}
        \rightTestChi{5}{$\chi^2$ Label}{P-Value}
        \drawDist;
      \end{axis}
    \end{tikzpicture}
  \end{ChiDist} &
  \begin{BinomDist}{10}{0.2}
    \begin{tikzpicture}[scale=0.6]
      \begin{axis}[DistAxis]
        \rightTail{3}{3}{0.2}{}
        \drawDist;
      \end{axis}
    \end{tikzpicture}
  \end{BinomDist} \\  \tabucline{-}
\end{tabu}
  
\vspace*{10pt}

\tableofcontents

\pagebreak

\section{Declaring a Distribution}

To start an illustration, declare the type of distribution that will
be used.  For the vast majority of the illustrations, only a single
type of distribution is used in any given illustration.  The
distributions that are supported are shown below.

\subsection{Normal Distributions} 

The declaration of a normal distribution has no parameters.

  \begin{minipage}{0.3\textwidth}
  \begin{lstlisting}
    \begin{NormDist}
      ...
    \end{NormDist}
  \end{lstlisting}
  \end{minipage}
  \begin{minipage}{0.6\textwidth}
  This sets up a standard normal distribution, though through
  labelling, it can illustrate non-standard normal distributions as
  well.  \textit{Of all of the distributions, the normal distribution
    is the least computationally burdensome for LaTex.}
  \end{minipage}
  
\subsection{Student T Distributions}

The declaration of a student T distribution requires setting the
number of degrees of freedom for the distribution.

  \begin{minipage}{0.3\textwidth}
  \begin{lstlisting}
    \begin{StudDist}{9}
      ...
    \end{StudDist}
  \end{lstlisting}
  \end{minipage}
  \begin{minipage}{0.6\textwidth}
    This creates a student T distribution.  The required parameter
    (``9'' in this example) is the number of degrees of freedom.
    \textit{Using this distribution is computationally burdensome
      for LaTex.}  For many illustrations, the difference between a
    student T distribution and a normal distribution is not noticable.
    Thus, unless there is a real need to display an accurate student T
    distribution, a ``fake'' student T distribution should be used
    instead.
  \end{minipage}

\subsection{Fake Student T Distributions}

The declaration of a fake student T distribution requires setting the
number of degrees of freedom for the distribution.

  \begin{minipage}{0.3\textwidth}
  \begin{lstlisting}
    \begin{FakeStudDist}{9}
      ...
    \end{FakeStudDist}
  \end{lstlisting}
  \end{minipage}
  \begin{minipage}{0.6\textwidth}
    A fake student T distribution uses a normal distribution for the
    curve, but labels the axis with a 't'.  The reason for this is
    that the normal distribution is much less computationally
    burdensome for LaTex.  Thus, documents which don't need the
    accuracy of the true student T will process much faster in LaTex
    if the fake student T distribution is used instead.  The parameter
    for the fake student T distribution (``9'' in this example) is not
    used, and exists solely to make switching back and forth between a
    computationally accurate rendition and the fake version simple.
  \end{minipage}

\subsection{$\chi^2$ Distributions}

The declaration of a $\chi^2$ distribution has no parameters.

  \begin{minipage}{0.3\textwidth}
  \begin{lstlisting}
    \begin{ChiDist}
      ...
    \end{ChiDist}
  \end{lstlisting}
  \end{minipage}
  \begin{minipage}{0.6\textwidth}
    This sets up a $\chi^2$ distribution.  The $\chi^2$ distribution
    does not support as many of the markup options as this distribution
    is asymmetric, with a tail on the right only.  I have yet to encounter
    a use for a left tail and as such, it is not (yet) supported.
  \end{minipage}

\subsection{Binomial Distributions}

The declaration of a binomial distribution requires setting number of
trials and the probability of success.

  \begin{minipage}{0.3\textwidth}
  \begin{lstlisting}
    \begin{BinomDist}{10}{0.2}
      ...
    \end{BinomDist}
  \end{lstlisting}
  \end{minipage}
  \begin{minipage}{0.6\textwidth}
    This creates a binomial distribution.  The first parameter
    (``10'' in this example) is the number of trials.  The second
    paramters (``0.2'' is this example) is the probability of
    success.
  \end{minipage}

\section{Adding Markups to Distributions}

The same basic commands are used to markup the distributions supported
by this package.  The process for marking up a normal distribution are
described below.  Then, special cases unique to particular
distributions are described.

For each of the markups, there is typically one parameter which
identifies the location of the markup, and a second parameter which
defines the label for the markup.  This allows for the case where you
want the item positioned at one place, but the label to say something
else.  For example, if the z-score is 5 (which would be beyond the end
of a typical drawing), you can position the item at z-score 4, but
have the label say 5.

\subsection{Summary of Markup Commands}

Shown below are example markups with text labels used in lieu of
numbers to identify each of the labels available in the markups.

\begin{minipage}{0.5\textwidth}
\begin{lstlisting}
\begin{NormDist}
  \begin{tikzpicture}[scale=0.7]
    \begin{axis}[DistAxis]
      \rightTail[fill=orange!50]{Z-Score}{Z Label}{Alpha}{Problem}
      \leftTail[fill=orange!50]{Z-Score}{Z Label}{Alpha}{Problem}
      \confidenceLevel[Zero]{Area}
      \secondAxis{Mean}{Label}
      \rightTestStatistic{Z-Score}{Z Label}{P-Value}{Problem}
      \leftTestStatistic{Z-Score}{Z Label}{P-Value}{Problem}
      \drawDist;
    \end{axis}
  \end{tikzpicture}
\end{NormDist}
\end{lstlisting}
\end{minipage}
\begin{minipage}{0.5\textwidth}
\begin{center}
\begin{NormDist}
  \begin{tikzpicture}[scale=0.7]
    \begin{axis}[DistAxis]
      \rightTail[fill=orange!50]{1.645}{Z Label}{Alpha}{Problem}
      \leftTail[fill=orange!50]{-1.645}{Z Label}{Alpha}{Problem}
      \confidenceLevel[Zero]{Area}
      \secondAxis{Mean}{Label}
      \rightTestStatistic{1.45}{Z Label}{P-Value}{Problem}
      \leftTestStatistic{-1.45}{Z Label}{P-Value}{Problem}
      \drawDist;
    \end{axis}
  \end{tikzpicture}
\end{NormDist}
\end{center}
\end{minipage}

In addition to leftTail and rightTail illustrations, there is
a centerFill command for situations where one or multiple regions
within the distribution are being illustrated.

\begin{minipage}{0.5\textwidth}
\begin{lstlisting}
\begin{NormDist}
  \begin{tikzpicture}[scale=0.7]
    \begin{axis}[DistAxis]
      \centerFill[fill=orange!50]
           {-1.645}{Left Z Label}{Left Problem}
           {1.645}{Right Z Label}{Right Problem}
           {Area}
      \confidenceLevel[Zero]{}
      \secondAxis{Mean}{Label}
      \rightTestStatistic{Z-Score}{Z Label}{P-Value}{Problem}
      \leftTestStatistic{Z-Score}{Z Label}{P-Value}{Problem}
      \drawDist;
    \end{axis}
  \end{tikzpicture}
\end{NormDist}
\end{lstlisting}
\end{minipage}
\begin{minipage}{0.5\textwidth}
\begin{center}
\begin{NormDist}
  \begin{tikzpicture}[scale=0.7]
    \begin{axis}[DistAxis]
      \centerFill[fill=orange!50]{-1.645}{Left Z Label}{Left Problem}
                                  {1.645}{Right Z Label}{Right Problem}{Area}
      \confidenceLevel[Zero]{}
      \secondAxis{Mean}{Label}
      \rightTestStatistic{1.45}{Z Label}{P-Value}{Problem}
      \leftTestStatistic{-1.45}{Z Label}{P-Value}{Problem}
      \drawDist;
    \end{axis}
  \end{tikzpicture}
\end{NormDist}
\end{center}
\end{minipage}

\subsection{Marking the Tails}

The tails of the distribution are identified with the \verb|\leftTail|
and/or \verb|\rightTail| commands.  Each of these commands has four
parameters: Value, Label, Alpha, and Problem.  The Value is the
standard normal z-score where the markup will be located.  The Label
is the text which will be printed at the base of the markup.  The
Alpha is the probability that is associated with the area in the tail.
The Problem is the text to put on the second axis.  The Label, Alpha,
and Problem parameters are optional in the sense that an empty value
will cause nothing to be printed.

In the example below, no second axis is shown, so this paramter is empty.  

\begin{minipage}{0.5\textwidth}
\begin{lstlisting}
\begin{NormDist}
  \begin{tikzpicture}[scale=0.7]
    \begin{axis}[DistAxis]
      \leftTail{-1.645}{-1.645}{0.05}{}
      \confidenceLevel{}
      \drawDist;
    \end{axis}
  \end{tikzpicture}
\end{NormDist}
\end{lstlisting}
\end{minipage}
\begin{minipage}{0.5\textwidth}
\begin{center}
\begin{NormDist}
  \begin{tikzpicture}[scale=0.7]
    \begin{axis}[DistAxis]
      \leftTail{-1.645}{-1.645}{0.05}{}
      \confidenceLevel{}
      \drawDist;
    \end{axis}
  \end{tikzpicture}
\end{NormDist}
\end{center}
\end{minipage}

The right tail is marked in a similar manner.

\begin{minipage}{0.5\textwidth}
\begin{lstlisting}
\begin{NormDist}
  \begin{tikzpicture}[scale=0.7]
    \begin{axis}[DistAxis]
      \rightTail{1.645}{1.645}{0.05}{}
      \confidenceLevel{}
      \drawDist;
    \end{axis}
  \end{tikzpicture}
\end{NormDist}
\end{lstlisting}
\end{minipage}
\begin{minipage}{0.5\textwidth}
\begin{center}
\begin{NormDist}
  \begin{tikzpicture}[scale=0.7]
    \begin{axis}[DistAxis]
      \rightTail{1.645}{-1.645}{0.05}{}
      \confidenceLevel{}
      \drawDist;
    \end{axis}
  \end{tikzpicture}
\end{NormDist}
\end{center}
\end{minipage}

When both tails need to be marked, both \verb|\leftTail| and
\verb|\rightTail| can be used as in the next example.

\begin{minipage}{0.5\textwidth}
\begin{lstlisting}
\begin{NormDist}
  \begin{tikzpicture}[scale=0.7]
    \begin{axis}[DistAxis]
      \leftTail{-1.645}{-1.645}{0.05}{}
      \rightTail{1.645}{1.645}{0.05}{}
      \confidenceLevel{}
      \drawDist;
    \end{axis}
  \end{tikzpicture}
\end{NormDist}
\end{lstlisting}
\end{minipage}
\begin{minipage}{0.5\textwidth}
\begin{center}
\begin{NormDist}
  \begin{tikzpicture}[scale=0.7]
    \begin{axis}[DistAxis]
      \leftTail{-1.645}{-1.645}{0.05}{}
      \rightTail{1.645}{-1.645}{0.05}{}
      \confidenceLevel{}
      \drawDist;
    \end{axis}
  \end{tikzpicture}
\end{NormDist}
\end{center}
\end{minipage}

\subsection{Marking the Center}

A typical usage for the centerFill command would be in illustrating
a confidence interval.  An example with typical parameters is shown.

\begin{minipage}{0.5\textwidth}
\begin{lstlisting}
\begin{NormDist}
  \begin{tikzpicture}[scale=0.7]
    \begin{axis}[DistAxis]
      \centerFill{-1.645}{-1.645}{}{1.645}{1.645}{}{}
      \confidenceLevel{0.90}
      \drawDist;
    \end{axis}
  \end{tikzpicture}
\end{NormDist}
\end{lstlisting}
\end{minipage}
\begin{minipage}{0.5\textwidth}
\begin{center}
\begin{NormDist}
  \begin{tikzpicture}[scale=0.7]
    \begin{axis}[DistAxis]
      \centerFill{-1.645}{-1.645}{}{1.645}{1.645}{}{}
      \confidenceLevel{0.90}
      \drawDist;
    \end{axis}
  \end{tikzpicture}
\end{NormDist}
\end{center}
\end{minipage}

\subsection{Marking the Test Statistic}

Similar to marking the tails, the \verb|\leftTestStatistic| and
\verb|\rightTestStatistic| are used to mark the location of the test
statistics. Each of these commands has four parameters: Value, Label,
P-Value, and Problem.  The Value is the standard normal z-score where
the markup will be located.  The Label is the text which will be
printed at the base of the markup.  The P-Value is the probability
that is associated with the area to the left or right of the test
statistic.  The Problem is the text to put on the second axis.  The
Label, P-Value, and Problem parameters are optional in the sense that an
empty value will cause nothing to be printed.

In the example below, no second axis is shown, so this parameter is empty.  

\begin{minipage}{0.5\textwidth}
\begin{lstlisting}
\begin{NormDist}
  \begin{tikzpicture}[scale=0.7]
    \begin{axis}[DistAxis]
      \confidenceLevel{}
      \leftTestStatistic{Z-Score}{Z Label}{P-Value}{}
      \drawDist;
    \end{axis}
  \end{tikzpicture}
\end{NormDist}
\end{lstlisting}
\end{minipage}
\begin{minipage}{0.5\textwidth}
\begin{center}
\begin{NormDist}
  \begin{tikzpicture}[scale=0.7]
    \begin{axis}[DistAxis]
      \confidenceLevel{}
      \leftTestStatistic{-1.45}{Z Label}{P-Value}{}
      \drawDist;
    \end{axis}
  \end{tikzpicture}
\end{NormDist}
\end{center}
\end{minipage}

Multiple test statistics can be included.  While the library tries
to put the probability at different heights, coding all possible
cases is complicated.  As a result, there is an optional parameter
that offsets the probability location if needed.

\begin{minipage}{0.5\textwidth}
\begin{lstlisting}
\begin{NormDist}
  \begin{tikzpicture}[scale=0.7]
    \begin{axis}[DistAxis]
      \leftTestStatistic{-2.0}{-2.0}{0.0228}{}
      \leftTestStatistic{-1.0}{-1.0}{0.1587}{}
      \leftTestStatistic{0}{0}{0.5}{}
      \leftTestStatistic[0.15]{1.0}{1.0}{0.8413}{}
      \leftTestStatistic[0.225]{2.0}{2.0}{0.9772}{}
      \drawDist;
    \end{axis}
  \end{tikzpicture}
\end{NormDist}
\end{lstlisting}
\end{minipage}
\begin{minipage}{0.5\textwidth}
\begin{center}
\begin{NormDist}
  \begin{tikzpicture}[scale=0.7]
    \begin{axis}[DistAxis]
      \leftTestStatistic{-2.0}{-2.0}{0.0228}{}
      \leftTestStatistic{-1.0}{-1.0}{0.1587}{}
      \leftTestStatistic{0}{0}{0.5}{}
      \leftTestStatistic[0.15]{1.0}{1.0}{0.8413}{}
      \leftTestStatistic[0.225]{2.0}{2.0}{0.9772}{}
      \drawDist;
    \end{axis}
  \end{tikzpicture}
\end{NormDist}
\end{center}
\end{minipage}

Right test statistics are done similarly.

\begin{minipage}{0.5\textwidth}
\begin{lstlisting}
\begin{NormDist}
  \begin{tikzpicture}[scale=0.7]
    \begin{axis}[DistAxis]
      \rightTestStatistic{-2.0}{-2.0}{0.0228}{}
      \rightTestStatistic{-1.0}{-1.0}{0.1587}{}
      \rightTestStatistic{0}{0}{0.5}{}
      \rightTestStatistic[0.15]{1.0}{1.0}{0.8413}{}
      \rightTestStatistic[0.225]{2.0}{2.0}{0.9772}{}
      \drawDist;
    \end{axis}
  \end{tikzpicture}
\end{NormDist}
\end{lstlisting}
\end{minipage}
\begin{minipage}{0.5\textwidth}
\begin{center}
\begin{NormDist}
  \begin{tikzpicture}[scale=0.7]
    \begin{axis}[DistAxis]
      \rightTestStatistic[0.225]{-2.0}{-2.0}{0.9772}{}
      \rightTestStatistic[0.15]{-1.0}{-1.0}{0.8413}{}
      \rightTestStatistic{0}{0}{0.5}{}
      \rightTestStatistic{1.0}{1.0}{0.1587}{}
      \rightTestStatistic{2.0}{2.0}{0.0228}{}
      \drawDist;
    \end{axis}
  \end{tikzpicture}
\end{NormDist}
\end{center}
\end{minipage}

\subsection{Normal vs Standard Normal Distributions}

For my students, instead of rescaling the z axis to match the problem,
I always draw a standard normal distribution ($\mu=0$, $\sigma=1$),
and then have a second axis which is matched to the units of the problem.
Especially in the introductory classes to the normal distribution, there
are times where no ``units of the problem'' exist, and so just want the
plain distribution shown below (not including the secondAxis and leaving
the {Problem} parameters to leftTail and leftTestStatistic effectively cause
the problem axis not to be drawn).

\begin{minipage}{0.5\textwidth}
\begin{lstlisting}
\begin{NormDist}
  \begin{tikzpicture}[scale=0.7]
    \begin{axis}[DistAxis]
      \leftTail{-1.645}{-1.645}{0.05}{}
      \confidenceLevel{}
      \leftTestStatistic{-1.45}{-1.45}{0.0735}{}
      \drawDist;
    \end{axis}
  \end{tikzpicture}
\end{NormDist}
\end{lstlisting}
\end{minipage}
\begin{minipage}{0.5\textwidth}
\begin{center}
\begin{NormDist}
  \begin{tikzpicture}[scale=0.7]
    \begin{axis}[DistAxis]
      \leftTail{-1.645}{-1.645}{0.05}{}
      \confidenceLevel{}
      \leftTestStatistic{-1.45}{-1.45}{0.0735}{}
      \drawDist;
    \end{axis}
  \end{tikzpicture}
\end{NormDist}
\end{center}
\end{minipage}

The associated right tail drawings without a secondary problem axis can
also be drawn.

\begin{minipage}{0.5\textwidth}
\begin{lstlisting}
\begin{NormDist}
  \begin{tikzpicture}[scale=0.7]
    \begin{axis}[DistAxis]
      \rightTail{1.645}{1.645}{0.05}{}
      \confidenceLevel{}
      \rightTestStatistic{1.45}{1.45}{0.0735}{}
      \drawDist;
    \end{axis}
  \end{tikzpicture}
\end{NormDist}
\end{lstlisting}
\end{minipage}
\begin{minipage}{0.5\textwidth}
\begin{center}
\begin{NormDist}
  \begin{tikzpicture}[scale=0.7]
    \begin{axis}[DistAxis]
      \rightTail{1.645}{1.645}{0.05}{}
      \confidenceLevel{}
      \rightTestStatistic{1.45}{1.45}{0.0735}{}
      \drawDist;
    \end{axis}
  \end{tikzpicture}
\end{NormDist}
\end{center}
\end{minipage}

Here is an example of a failing left tail test hypothesis test with
typical numbers filled in.  Right tailed tests are done similarly.

\begin{minipage}{0.5\textwidth}
\begin{lstlisting}
\begin{NormDist}
  \begin{tikzpicture}[scale=0.7]
    \begin{axis}[DistAxis]
      \leftTail{-1.645}{-1.645}{0.05}{}
      \confidenceLevel{}
      \secondAxis{0.35}{}
      \leftTestStatistic{-1.45}{-1.45}{0.0735}{0.37}
      \drawDist;
    \end{axis}
  \end{tikzpicture}
\end{NormDist}
\end{lstlisting}
\end{minipage}
\begin{minipage}{0.5\textwidth}
\begin{center}
\begin{NormDist}
  \begin{tikzpicture}[scale=0.7]
    \begin{axis}[DistAxis]
      \leftTail{-1.645}{-1.645}{0.05}{}
      \confidenceLevel{}
      \secondAxis{0.35}{}
      \leftTestStatistic{-1.45}{-1.45}{0.0735}{0.37}
      \drawDist;
    \end{axis}
  \end{tikzpicture}
\end{NormDist}
\end{center}
\end{minipage}

While the package knows how to compute student T distributions for
various degrees of freedom, its not really clear that this is useful.
In particular, since multiple distributions are not included on
the same drawing, there is no way for the reader to know whether or
not a Student T or Normal distribution is actually being drawn!
Further, since the Student T is much more computationally intensive
to draw, there is a definite advantage to drawing Normal distributions
even for the Student T case.

\begin{minipage}{0.5\textwidth}
\begin{lstlisting}
\begin{StudDist}{9}
  \begin{tikzpicture}[scale=0.7]
    \begin{axis}[DistAxis]
      \centerFill{-1.645}{-1.645}{30}{1.645}{1.645}{50}{}
      \confidenceLevel{0.90}
      \secondAxis{40}{lbs}
      \drawDist;
    \end{axis}
  \end{tikzpicture}
\end{StudDist}
\end{lstlisting}
\end{minipage}
\begin{minipage}{0.5\textwidth}
\begin{center}
\begin{StudDist}{9}
  \begin{tikzpicture}[scale=0.7]
    \begin{axis}[DistAxis]
      \centerFill{-1.645}{-1.645}{30}{1.645}{1.645}{50}{}
      \confidenceLevel{0.90}
      \secondAxis{40}{lbs}
      \drawDist;
    \end{axis}
  \end{tikzpicture}
\end{StudDist}
\end{center}
\end{minipage}

For $\chi^2$ distributions, the notion of a problem axis
doesn't make sense for the types of problems I use.  Currently,
support for a problem axis is not present.

\begin{minipage}{0.5\textwidth}
  \begin{lstlisting}
    \begin{ChiDist}
    \begin{tikzpicture}[scale=0.7]
    \begin{axis}[DistAxis]
      \rightTailChi[fill=orange!50]{4}{$\chi^2$ Label}{Alpha}
      \rightTestChi{5}{$\chi^2$ Label}{P-Value}
      \drawDist;
    \end{axis}
  \end{tikzpicture}
    \end{ChiDist}
  \end{lstlisting}
\end{minipage}
\begin{minipage}{0.5\textwidth}
\begin{center}
\begin{ChiDist}
  \begin{tikzpicture}[scale=0.7]
    \begin{axis}[DistAxis]
      \rightTailChi[fill=orange!50]{4}{$\chi^2$ Label}{Alpha}
      \rightTestChi{5}{$\chi^2$ Label}{P-Value}
      \drawDist;
    \end{axis}
  \end{tikzpicture}
\end{ChiDist}
\end{center}
\end{minipage}

Finally, the library also has support for binomial distributions.

\begin{center}
\begin{BinomDist}{10}{0.2}
  \begin{tikzpicture}[scale=0.7]
    \begin{axis}[DistAxis]
      \rightTail{3}{3}{0.2}{}
      \drawDist;
    \end{axis}
  \end{tikzpicture}
\end{BinomDist}
\end{center}  

\begin{center}
\begin{BinomDist}{8}{0.5}
  \begin{tikzpicture}[scale=0.7]
    \begin{axis}[DistAxis]
      \leftTail{3}{3}{0.2}{}
      \rightTail{6}{6}{0.2}{}
      \drawDist;
    \end{axis}
  \end{tikzpicture}
\end{BinomDist}
\end{center}  

\section{Real Usage Examples}

Find the probability that a z score will be above -1. 

\begin{center}
  \begin{NormDist}
    \begin{tikzpicture}[scale=0.9]
      \begin{axis}[DistAxis]
	\rightTail{-1.0}{-1.0}{}{}
        \confidenceLevel{}
        \draw  node at (axis cs:-2.5, 0.3) [text width=3cm] {Use z=-1.00 in Table to find this area.};
	\draw  node at (axis cs:0.3, 0.1)  [text width=2cm] {What is this area?};

        \draw    [thick, -latex] (axis cs:-2.4, 0.2) -- (axis cs:-2.0, 0.1);
        \drawDist;
      \end{axis}
    \end{tikzpicture}
  \end{NormDist}
\end{center}

  Tall Clubs International has a requirement that women must be at
  least 70 inches tall.  Given that women have normally distributed
  heights with a mean of 63.8 inches and a standard deviation of 2.6
  inches, find the percentage of women who satisfy that height
  requirement.

\begin{minipage}{0.5\textwidth}
  \begin{lstlisting}
    \begin{NormDist}
      \begin{tikzpicture}[scale=0.6]
        \begin{axis}[DistAxis]
          \rightTail{2.38}{2.38}{0.0087}{70}
          \confidenceLevel{}
          \secondAxis{63.8}{x (in)}
          \drawDist;
        \end{axis}
      \end{tikzpicture}
    \end{NormDist}
  \end{lstlisting}
\end{minipage}
\begin{minipage}{0.5\textwidth}
\begin{center}
  \begin{NormDist}
    \begin{tikzpicture}[scale=0.6]
      \begin{axis}[DistAxis]
        \rightTail{2.38}{$2.38$}{0.0087}{70}
        \confidenceLevel{}
        \secondAxis{63.8}{x (in)}
        \drawDist;
      \end{axis}
    \end{tikzpicture}
  \end{NormDist}
\end{center}
\end{minipage}

\section{Special Situations}

Beyond simply illustrating problems, this package can be used
to create illustrations useful in teaching statistical concepts.
This section has examples of these kinds of illustrations.

\subsection{Student T vs Normal Distributions}

An application where drawing the true Student T distribution is
meaningful is as a comparison of the impact of degrees of freedom on
the curve, as well as a comparison between Student T and Normal such
as the figure below.  As shown, the student T distribution approaches
a normal distribution.  By the time degrees of freedom hits 50, the
differences between the distributions are not visible at this scale.

\begin{minipage}{0.5\textwidth}
\begin{lstlisting}
\begin{NormDist}
  \begin{tikzpicture}[scale=0.7]
    \begin{axis}[DistAxis]

    \addplot [very thick,draw=red!80!black] {std_stud(1,x)};
    \addplot [very thick,draw=red!80!black] {std_stud(2,x)};
    \addplot [very thick,draw=red!80!black] {std_stud(5,x)};
    \addplot [very thick,draw=red!80!black] {std_stud(50,x)};
    \addplot [very thick,draw=blue!80!black] {std_norm(x)};
    ...
    \end{axis}
  \end{tikzpicture}
\end{NormDist}
\end{lstlisting}
\end{minipage}
\begin{minipage}{0.5\textwidth}
\begin{center}
\begin{NormDist}
  \begin{tikzpicture}[scale=0.7]
    \begin{axis}[DistAxis]

    \addplot [very thick,draw=red!80!black] {std_stud(1,x)};
    \addplot [very thick,draw=red!80!black] {std_stud(2,x)};
    \addplot [very thick,draw=red!80!black] {std_stud(5,x)};
    \addplot [very thick,draw=red!80!black] {std_stud(50,x)};

    \draw    [anchor=west] node at (axis cs:1.5, 0.20) { df=1 };
    \draw    [anchor=west] node at (axis cs:1.5, 0.25) { df=2 };
    \draw    [anchor=west] node at (axis cs:1.5, 0.30) { df=5 };
    \draw    [anchor=west] node at (axis cs:1.5, 0.35) { df=50 };

    \draw    [thick, -latex] (axis cs:1.4, 0.20) -- (axis cs:0.8, {std_stud(1,0.8)});
    \draw    [thick, -latex] (axis cs:1.4, 0.25) -- (axis cs:0.75, {std_stud(2,0.75)});
    \draw    [thick, -latex] (axis cs:1.4, 0.30) -- (axis cs:0.7, {std_stud(5,0.7)});
    \draw    [thick, -latex] (axis cs:1.4, 0.35) -- (axis cs:0.65, {std_stud(50,0.65)});

    \addplot [very thick,draw=blue!80!black] {std_norm(x)};

    \draw    [anchor=east] node at (axis cs:-1.5, 0.35) { norm };
    \draw    [thick, -latex] (axis cs:-1.4, 0.35) -- (axis cs:-0.5, {std_norm(-0.5)});

    \draw    [thick,magenta] (axis cs:0, -0.02) -- (axis cs:0, 0.02);
    \draw    node at (axis cs:0,-0.02) [below] {0};

    \end{axis}
  \end{tikzpicture}
\end{NormDist}
\end{center}
\end{minipage}

\subsection{Binomial vs Normal Distributions}

Shown below is another way of combining multiple distributions into a
single illustration.  Note, here, how the order of the tikzpicture and
NormDist has been swapped.  This picture shows how a binomial
distribution compares with a normal distribution.  In the situation
shown with $n=8$ and $p=0.4$, $np(1-p) < 10$ meaning that a normal
distribution is not yet a good approximateion for the binomial
distribution.

\begin{minipage}{0.5\textwidth}
\begin{lstlisting}
\begin{tikzpicture}[scale=0.7]
  \begin{NormDist}
    \begin{axis}[DistAxis, domain=-2.67:2.38]
    \addplot [very thick,draw=red!80!black] {std_norm(x)};
    \end{axis}
  \end{NormDist}
  \begin{BinomDist}{8}{0.4}
    \begin{axis}[DistAxis]
    \addplot [very thick,draw=blue!80!black] {binom(x,8,0.4)};
    \end{axis}
  \end{BinomDist}
\end{tikzpicture}
\end{lstlisting}
\end{minipage}
\begin{minipage}{0.5\textwidth}
\begin{center}
\begin{tikzpicture}[scale=0.7]
  \begin{NormDist}
% z0 = (-0.5 - (8*0.4)) / (sqrt(8*0.4*0.6))
% z1 = (6.5 - (8*0.4)) / (sqrt(8*0.4*0.6))
    \begin{axis}[DistAxis, domain=-2.67:2.38]
    \addplot [very thick,draw=red!80!black] {std_norm(x)};
    \end{axis}
  \end{NormDist}
  \begin{BinomDist}{8}{0.4}
    \begin{axis}[DistAxis]
    \addplot [very thick,draw=blue!80!black] {binom(x,8,0.4)};
    \end{axis}
  \end{BinomDist}
\end{tikzpicture}
\end{center}
\end{minipage}

\subsection{Empirical Rule}

Early in an introductory statistics class, when the normal distribution
is being introduced to the class, the Empirical Rule is used to teach
basic properties of the normal distribution.

\begin{center}
  \begin{NormDist}
    \begin{tikzpicture}[scale=0.7]
      \EmpiricalRule
    \end{axis}
  \end{tikzpicture}
\end{NormDist}
\end{center}

\subsection{Confidence Intervals}

\begin{center}
\begin{NormDist}
  \begin{tikzpicture}[scale=0.7]
     \ConfidenceInterval
     \end{axis}
   \end{tikzpicture}
 \end{NormDist}
\end{center}

\subsection{Hypothesis Testing}

\begin{center}
  \begin{NormDist}
    \begin{tikzpicture}[scale=0.7]
      \RightTailHypothesis
    \end{axis}
  \end{tikzpicture}
\end{NormDist}
\end{center}


\end{document}
%%%%%%%%%%%%%%%%%%%%%%%%%%%%%%%%%%%%%%%%%%%%%%%%%%%%%%%%%%%%%%%%%%%%%%%%%%%%%%
%%%%%%%%%%%%%%%%%%%%%%%%%%%%%%%%%%%%%%%%%%%%%%%%%%%%%%%%%%%%%%%%%%%%%%%%%%%%%%

\begin{minipage}{0.45\textwidth}
  \textsl{The reading speed of sixth-grade students is normally distributed
  with a mean speed of 125 words per minute and a standard deviation
  of 24 words per minute.  Find the probability that a randomly
  selected student reads between 110 and 150 words per minute.}


\end{minipage}
\hfill
\begin{minipage}{0.45\textwidth}
  \textsl{Most IQ tests have a mean of 100 and standard deviation of
    15.  Using such a test, a random sample of thirty students have a
    mean score of 112.  Are the students above average?}
  
  \begin{center}
    \begin{NormDist}
      \begin{tikzpicture}[scale=0.6]
        \begin{axis}[DistAxis]
          \rightTail{1.96}{1.96}{0.025}{}
          \rightTestStatistic{0.8}{0.8}{0.21}{112}
          \confidenceLevel{}
          \secondAxis{100}{pts} % mean
          \drawDist;
        \end{axis}
      \end{tikzpicture}
    \end{NormDist}
  \end{center}
\end{minipage}
