\usepackage{tabularray}
\usetikzlibrary{shapes.arrows, fadings, calc} % Needed by ConfidenceInterval

%%%%%%%%%%%%%%%%%%%%%%%%%%%%%%%%%%%%%%%%%%%%%%%%%%%%%%%%%%%%%%%%%%%%%%%%%%%
% Critical Values and Correlation Coefficients for Linear Regression.
%
% \rvalueGraph{ CV } {r} {neg CV Label}{pos CV Label}{r Label}
% \rValueGraph{ 0.533}{ -0.958}{$-0.533$}{$0.533$}{$-0.958$}

\newcommand{\rValueGraph}[5] {
  \begin{tikzpicture}[scale=1.0,
                      every text node part/.style={align=center,text width=2cm,font=\small}]
    \begin{axis}[
       axis y line=none,
       axis x line=middle,
       x axis line style=-,
       height=6cm,
       width=9cm,
       xmin=-1.01,
       xmax=1.01,
       ymin=-1,
       ymax=1.0,
       xtick={-1, 0, 1},
       every x tick/.style={ black, thick},
       clip=false
       ]

%    \addplot [draw=none]{0}; % Needed to make the axis work correctly if no drawing.!

    \fill [pattern=north east lines,pattern color=green] (axis cs:-1.0,0) rectangle (axis cs:-#1,0.2);
    \fill [pattern=north east lines,pattern color=red!50!white] (axis cs:-#1,0) rectangle (axis cs:#1,0.2);
    \fill [pattern=north east lines,pattern color=green] (axis cs:#1,0) rectangle (axis cs: 1.0,0.2);

    \addplot [mark=none] coordinates {(-1.0,0) (-1.0,0.2)};
    \addplot [red, mark=none] coordinates {(-#1,0) (-#1,0.2)};
    \addplot [red, mark=none] coordinates {(#1,0)  (#1,0.2)};
    \addplot [mark=none] coordinates {(1.0,0)  (1.0,0.2)};

    \ifx&#2&%
    \else
      \addplot [mark=none, very thick] coordinates {(#2,-0) (#2,0.25)};
      \node at (axis cs:#2,0.31){ #5 };
    \fi

    \node at (axis cs:-#1,-0.3){ #3 };
    \node at (axis cs:#1,-0.3){ #4 };

    \node at (axis cs:-0.7,0.6){Negative Correlation};
    \node at (axis cs:0.0,0.6){Weak or No Linear Relation};
    \node at (axis cs:0.7,0.6){Positive Correlation};

    \end{axis}
  \end{tikzpicture}
}

%%%%%%%%%%%%%%%%%%%%%%%%%%%%%%%%%%%%%%%%%%%%%%%%%%%%%%%%%%%%%%%%%%%%%
% Empirical Rule -- Annotate a normal curve with the percentages for
% each std dev.
%
%% \begin{center}
%% \begin{NormDist}
%%   \begin{tikzpicture}[scale=0.7]
%%     \EmpiricalRule
%%     \end{axis}
%%   \end{tikzpicture}
%% \end{NormDist}
%% \end{center}

\newcommand{\EmpiricalRule} {
  \begin{axis}[DistAxis,
               xtick={-3,-2,-1,0,1,2,3}, 
               every x tick/.style={color=black, thick},
              ]

    \drawDist;
      
    \draw [thick] (axis cs:-3,0) -- (axis cs:-3,{std_norm(-3)});
    \draw [thick] (axis cs:-2,0) -- (axis cs:-2,{std_norm(-2)});
    \draw [thick] (axis cs:-1,0) -- (axis cs:-1,{std_norm(-1)});
    \draw [thick,magenta] (axis cs:0,0) -- (axis cs:0,{std_norm(0)});
    \draw [thick] (axis cs:1,0) -- (axis cs:1,{std_norm(1)});
    \draw [thick] (axis cs:2,0) -- (axis cs:2,{std_norm(2)});
    \draw [thick] (axis cs:3,0) -- (axis cs:3,{std_norm(3)});

    \node at (axis cs:3.5,0.1)   {\footnotesize 0.15\%};
    \node at (axis cs:2.5,0.1)   {\footnotesize 2.35\%};
    \node at (axis cs:1.5,0.05)  {\footnotesize 13.5\%};
    \node at (axis cs:0.5,0.1)   {\footnotesize 34\%};
    \node at (axis cs:-0.5,0.1)  {\footnotesize 34\%};
    \node at (axis cs:-1.5,0.05) {\footnotesize 13.5\%};
    \node at (axis cs:-2.5,0.1)  {\footnotesize 2.35\%};
    \node at (axis cs:-3.5,0.1)  {\footnotesize 0.15\%};

    \draw [thick] (axis cs:-1,0.53) -- (axis cs:-1,{std_norm(-1)+0.03});
    \draw [thick] (axis cs:1,0.53) -- (axis cs:1,{std_norm(1)+0.03});
    \draw [latex-latex](axis cs:-1,0.5) -- node [fill=white] {$68\%$} (axis cs:1,0.5);

    \draw [thick] (axis cs:-2,0.63) -- (axis cs:-2,{std_norm(-2)+0.03});
    \draw [thick] (axis cs:2,0.63) -- (axis cs:2,{std_norm(2)+0.03});
    \draw [latex-latex](axis cs:-2,0.6) -- node [fill=white] {$95\%$} (axis cs:2,0.6);

    \draw [thick] (axis cs:-3,0.73) -- (axis cs:-3,{std_norm(-3)+0.03});
    \draw [thick] (axis cs:3,0.73) -- (axis cs:3,{std_norm(3)+0.03});
    \draw [latex-latex](axis cs:-3,0.7) -- node [fill=white] {$99.7\%$} (axis cs:3,0.7);

% The \end{axis} happens in the code after any other drawing on this axis.
}

\newcommand{\ConfidenceInterval}[1][Confidence Interval] {
  \begin{axis}[DistAxis]
    \centerFill{-1.75}{}{}{1.75}{}{}{}

    \leftTail[fill=red!20]{-1.75}{}{}{}
    \rightTail[fill=red!20]{1.75}{}{}{}

    \confidenceLevel[\hbox{z=0}]{}
    \drawDist;

    \draw [thick] (axis cs:-1.75,0) -- (axis cs:-1.75,{std_norm(-1.75)}) coordinate[pos=1] (L);
    \draw [thick] (axis cs:1.75,0) -- (axis cs:1.75,{std_norm(1.75)}) coordinate[pos=1] (R);

    \draw [thick] (axis cs:-1.75,0.53) -- (axis cs:-1.75,{std_norm(-1.75)+0.03});
    \draw [thick] (axis cs:1.75,0.53) -- (axis cs:1.75,{std_norm(1.75)+0.03});

    \path let \p1=($(R)-(L)$) in (axis cs:0, 0.5)
    node [draw=gray, 
      left color=white,
      right color=green!30,
      double arrow,
      minimum height=\x1,
      minimum width=1cm,
      double arrow head extend=0.05cm,inner xsep=0pt
    ] {#1};

% The \end{axis} happens in the code after any other drawing on this axis.
}

\tikzset{
   stat lib fat arrow/.style 2 args={
    every to/.style={
      to path={
        let \p1 = ($(\tikztotarget)-(\tikztostart)$),
            \n1 = {int(mod(scalar(atan2(\y1,\x1))+360, 360))}, % calculate angle in range [0,360)
            \p2 = ($(\tikztotarget.\n1+180)-(\tikztostart.\n1)$),
            \n2 = {veclen(\x2,\y2)}
        in
        -- (\tikztotarget)
        node[draw, % blue,
             #1,   % pass color stuff here!
             inner xsep=0pt,inner ysep=7pt, % use inner ysep to set width
             minimum height=\n2-\pgflinewidth+4pt,
             single arrow,
             single arrow head extend=2.0pt,
             rotate=\n1, % not shape border rotate, because that for some reason didn't work
             anchor=tip, % anchor=tip added, pos=0.5 removed
             #2          % arguments passed to fat arrow added here
             ]
          at (\tikztotarget.\n1+180)
          {} \tikztonodes}
  }},
  stat lib fat arrow/.default= % empty default for argument of fat arrow
}

\tikzset{
   stat lib no arrow/.style 2 args={
    every to/.style={
      to path={
        let \p1 = ($(\tikztotarget)-(\tikztostart)$),
            \n1 = {int(mod(scalar(atan2(\y1,\x1))+360, 360))}, % calculate angle in range [0,360)
            \p2 = ($(\tikztotarget.\n1+180)-(\tikztostart.\n1)$),
            \n2 = {veclen(\x2,\y2)}
        in
        -- (\tikztotarget)
        node[draw, % blue,
             #1,   % pass color stuff here!
             inner xsep=0pt,inner ysep=7pt, % use inner ysep to set width
             minimum height=\n2-\pgflinewidth+4pt,
             single arrow,
             single arrow head extend=2.0pt,
             rotate=\n1, % not shape border rotate, because that for some reason didn't work
             anchor=tip, % anchor=tip added, pos=0.5 removed
             #2          % arguments passed to no arrow added here
             ]
          at (\tikztotarget.\n1+180)
          {} \tikztonodes}
  }},
  stat lib no arrow/.default= % empty default for argument of no arrow
}

\newcommand{\RightTailHypothesis}{
  \begin{axis}[DistAxis]
    \centerFill{-3.0}{}{}{1.75}{}{}{}

    \rightTail[fill=red!20]{1.75}{}{$\alpha$}{}

    \confidenceLevel[0]{}
    \drawDist;

    \node at (axis cs:-4.0,0.5)   (A) {};
    \node at (axis cs:1.75,0.5) (B) {};
    \node at (axis cs:4.0,0.5)  (C) {};

    \path[stat lib fat arrow={green!50!black,left color = green!30, right color = white}{}] (B) to (A);
    \path[stat lib fat arrow={red!50!black, left color = white, right color = red!30}{}] (B) to (C);

    \path (A) -- (B) node [midway] {Expected};
    \path (B) -- (C) node [midway] {Unusual};

    \draw [thick] (axis cs:1.75,0)    -- (axis cs:1.75,{std_norm(1.75)});
    \draw [thick] (axis cs:1.75,0.53) -- (axis cs:1.75,{std_norm(1.75)+0.03});
}

\newcommand{\LeftTailHypothesis}{
  \begin{axis}[DistAxis]
    \centerFill{-1.75}{}{}{3.0}{}{}{}

    \leftTail[fill=red!20]{-1.75}{}{$\alpha$}{}

    \confidenceLevel[\hbox{z=0}]{}
    \drawDist;

    \node at (axis cs:4.0,0.5)   (A) {};
    \node at (axis cs:-1.75,0.5) (B) {};
    \node at (axis cs:-4.0,0.5)  (C) {};

    \path[stat lib fat arrow={green!50!black,left color = green!30, right color = white}{}] (B) to (A);
    \path[stat lib fat arrow={red!50!black, left color = white, right color = red!30}{}] (B) to (C);

    \path (A) -- (B) node [midway] {Expected};
    \path (B) -- (C) node [midway] {Unusual};

    \draw [thick] (axis cs:-1.75,0)    -- (axis cs:-1.75,{std_norm(-1.75)});
    \draw [thick] (axis cs:-1.75,0.53) -- (axis cs:-1.75,{std_norm(-1.75)+0.03});
}

\newcommand{\TwoTailHypothesis}{
  \begin{axis}[DistAxis]
    \centerFill{-1.75}{}{}{1.75}{}{}{}

    \leftTail[fill=red!20]{-1.75}{}{$\alpha/2$}{}
    \rightTail[fill=red!20]{1.75}{}{$\alpha/2$}{}

    \confidenceLevel[\hbox{z=0}]{}
    \drawDist;

    \node at (axis cs:4.0,0.5)   (A) {};
    \node at (axis cs:1.75,0.5)  (B) {};
    \node at (axis cs:-1.75,0.5) (C) {};
    \node at (axis cs:-4.0,0.5)  (D) {};

    \path[stat lib fat arrow={red!50!black,left color = red!30, right color = white}{}] (B) to (A);
    \path let \p1 = ($(B)-(C)$) in node (rect) at (axis cs:0, 0.5) [draw=green!50!black, minimum width=\x1, minimum height=14pt, fill=green!30] {};
    \path[stat lib fat arrow={red!50!black,left color = white, right color = red!30}{}] (C) to (D);

    \path (A) -- (B) node [midway] {Unusual};
    \path (B) -- (C) node [midway] {Expected};
    \path (C) -- (D) node [midway] {Unusual};

    \draw [thick] (axis cs:-1.75,0)    -- (axis cs:-1.75,{std_norm(-1.75)});
    \draw [thick] (axis cs: 1.75,0)    -- (axis cs: 1.75,{std_norm( 1.75)});
    \draw [thick] (axis cs:-1.75,0.53) -- (axis cs:-1.75,{std_norm(-1.75)+0.03});
    \draw [thick] (axis cs: 1.75,0.53) -- (axis cs: 1.75,{std_norm( 1.75)+0.03});
}

\newcommand{\rValues}{
  \begin{tblr}{colspec={cccl},
               rowspec={Q},
               row{even} = {black!20},
               row{1} = {white},
               row{2} = {red!20},
               hline{2}={solid},
               hline{3}={solid},
               hline{Z}={solid},
               vline{1}={2-Z}{solid},
               vline{Z}={2-Z}{solid}
  }
  \SetCell[c=4]{c} \textbf{Correlation Coefficient} & \\
  \SetCell[c=3]{c} \textbf{Interval} &&& \SetCell{l} \textbf{Meaning} \\
       & -1 &      & Exact negative linear relationship \\ 
  -1.0 & .. & -0.8  & Strong negative linear relationship \\
  -0.8 & .. & -0.5  & Moderate negative linear relationship \\
  -0.5 & .. & 0     & Weak negative linear relationship \\
       &  0 &      & No linear relationship \\
     0 & .. &  0.5 & Weak positive linear relationship \\
   0.5 & .. & 0.8  & Moderate positive linear relationship \\
   0.8 & .. & 1    & Strong positive linear relationship \\
       &  1 &      & Exact positive linear relationship \\
  \end{tblr}
}

\newcommand{\rSqValues}{
  \begin{tblr}{colspec={cccl},
               rowspec={Q},
               row{even} = {black!20},
               row{1} = {white},
               row{2} = {red!20},
               hline{2}={solid},
               hline{3}={solid},
               hline{Z}={solid},
               vline{1}={2-Z}{solid},
               vline{Z}={2-Z}{solid}
  }
  \SetCell[c=4]{c} \textbf{Coefficient of Determination} & \\
  \SetCell[c=3]{c} \textbf{Interval} &&& \SetCell{l} \textbf{Meaning} \\
       & 0   &      & the variables are unrelated \\
  0    & ..  & 0.5  & the variables are weakly related \\ 
  0.5  & ..  & 0.8  & the x-value can roughly predict y-values \\
  0.8  & ..  & 0.95 & the x-value can reasonably predict y-values \\
  0.95 & ..  & 1.0  & the x-value does a good job of predicting y-values \\
       & 1.0 &      & the x-value perfectly predicts y-values \\ 
  \end{tblr}
}


  %% \end{tabu}}
